\chapter{Introduction}

\label{Chapter1}

\lhead{Chapter 1. \emph{Introduction}}

% \emph{Context; motivation for the project; problem statement; outline of dissertation.}

The goal of the project in this thesis is to explore the viability of a new approach to user adaptation, where the requirements of the application are significantly more constraining than in most cases seen in previous academic work.

The application case is an anonymous, web-based video conferencing service called \url{appear.in}. The most important part of that description is the term ``anonymous'', as we not only need to deal with a complete lack of demographic information about the users -- we do not even have the ability to consistently identify them. Luckily, there are a few workarounds to the issue of identification that make the problem at least partly tractable.

Previous work in the area of user adaptation has often adhered to the concept of recommending some kind of resource -- often being reducible to the task of finding relevant items to sell, or interesting pages in a website hierarchy. The goal of this project, however, is not to \emph{find good things}, it is to \emph{optimize the user experience of an application}. More specifically, the goal is to be able to predict which version of the application will lead to better performance for each user, down to the individual feature level.

Let us start off with some background on why I believe this is an especially exciting time for user adaptation on the web, before section~\ref{sec:appearin_intro} introduces the specific application case of appear.in. Finally, section~\ref{sec:adaptation_intro} will provide an introduction to the realm of user adaptation.

\section{Background and motivation}
\label{sec:motivation}

The web is an interesting venue in which to explore new ways of user adaptation and personalization. Even today, when we are able to build almost every kind of application right in the web browser, a fundamental concept remains: there is still a server serving the application code every single time a user opens the site\footnote{Caching aside, of course.}.

\subsection{The modern web}
\label{sub:the_modern_web}

This is the modern web: a place where applications are nearing the power of traditional desktop applications, but with the potential of being continuously tailored to suit each user individually.

When we say that web applications are nearing the power of traditional desktop applications, we are of course talking about the introduction of HTML5. HTML was primarily designed as a language for semantically describing documents, and little more. The last decade, however, the concept of web applications has thoroughly established itself, while lacking clear standardization efforts from the W3C. The HTML5 specification is an attempt to remedy this~\cite{W3CHTML5_intro}, by providing standards and guidelines for the browser vendors on how to implement a wide range of common APIs.

In practice, these APIs lets websites do things like:

\begin{itemize}
    \item play audio and video\footnote{\url{http://dev.w3.org/html5/spec-author-view/video.html}}
    \item generate graphics\footnote{\url{http://www.w3.org/TR/2dcontext/}}
    \item access your webcam and microphone\footnote{\url{http://www.w3.org/TR/mediacapture-streams/}}
    \item handle and manipulate arbitrary files\footnote{\url{http://www.w3.org/TR/FileAPI/}}
    \item send and recieve data over full-duplex socket connections\footnote{\url{http://www.w3.org/TR/websockets/}}
\end{itemize}

...as well as a wide range of other things.

In general, HTML5 enables web applications to do most of the things one would need plugins or native applications for just a few years ago.

One of these new API specifications is called WebRTC\footnote{Web Real-Time Communication.}, and it is the last piece of the API puzzle enabling applications like appear.in.

\section{The application case}
\label{sec:appearin_intro}

For a long time, developing video conferencing services was an extremely challenging discipline, and as a result the market has consisted of a correspondingly low number of actors. However, this trend is currently in the process of being shaken and turned on its head with the introduction of HTML5; more specifically, with the introduction of the WebRTC specification.

\subsection{The enabling piece of technology: WebRTC}

As the name implies, WebRTC handles real-time communication, but for our case, the important aspect of the technology is that it is able to do so peer-to-peer. Although it per design is a protocol for exchanging arbitrary data between peers, it is particularly geared toward multimedia streams. For instance, the traditionally cumbersome task of setting up a two-way audiovisual connection is now a matter of dropping around 40 lines of boilerplate Javascript into a web page\footnote{For an excellent introduction, see: \url{http://www.html5rocks.com/en/tutorials/webrtc/basics/}.}.

Although the WebRTC specification at the time of writing still officially is a \emph{working draft} in the W3C\footnote{The latest specification can be found here: \url{http://dev.w3.org/2011/webrtc/editor/webrtc.html}.}, most of the large browser vendors have already implemented it. Consequentially, a plethora of applications leveraging this technology are already available, with ever more being launched every month.

\subsection{Introducing appear.in}

One of these applications is called appear.in, and it will serve as the main use case in this thesis. Like many other WebRTC applications, appear.in concerns itself with video conferencing.

The idea is simple enough: a conversation happens between users who are in the same room at the same time. The central idea, though, is that the \emph{conversation} is identified solely by the URL in use, and not in any way by the peers connecting. As an example, if any two people are visiting \url{https://appear.in/ntnu} at the same time, they will see and hear each other and can start chatting away without any further call setup or configuration.

This view of a conversation as not really being an enitity in its own right, but rather an \emph{effect} of people being in the same room at the same time, breaks with the traditional model of audiovisual communication. Traditionally, talking to someone not present has been a process of one person \emph{calling} the other one, with group conversations usually being nothing more than an extension of this concept. appear.in doesn't concern itself with distinguishing between callers and callees, has no simple concept of a ``conversation'', and generally does not enforce any particular way of using the service -- apart from requiring the conversation venue to be identifiable by a string of characters.

\subsection{Usage patterns}

Until the arrival of WebRTC, this particular way of thinking about conversations for anything but textual conversations hasn't been a big thing. However, the simplicity of the room concept opens the service up for a wide variety of uses\footnote{In addition to traditional video calls, we've already seen it used for everything from virtual offices and team meeting rooms to baby monitoring and remote tutoring, just to name a few.}. These wildly varied use cases are where the motivation for this project stems from:

\begin{enumerate}
  \item If the users' behaviors are quantified, will any clear and distinct usage patterns emerge?
  \item If so, can the different uses be better served by dynamically adapting the product to fit each of them?
\end{enumerate}

\subsection{Anonymity and privacy}
\label{sub:anonymity_privacy}

appear.in is an anonymous communication service. No personal information is ever collected about the users, and not even IP-addresses or geolocational data is logged on an individual level. By tracking individual \emph{browsers} using cookies, however, we can track users -- or more precisely, browsers -- over time.

By logging various events that we deem interesting along with a cookie value identifying the browser, we can reconstruct user sessions, and connect them to the application users pseudonomously. By ``pseudonomously'' it is meant that we identify the users solely by a random string set in their browser.

This all opens a wide series of questions bordering to sociological aspects of web usage:

\begin{enumerate}
  \item
    To what extent can an anonymous web service be adapted?
  \item
    Is user behavior enough to provide a satisfactory personalized user experience?
  % \item
  %   Will a personalized user experience go against the users' expectations of appear.in as an anonymous web service?
  % \item
  %   How can we measure any of this?
\end{enumerate}

These are issues which will be revisited throughout the thesis.

\subsubsection{Absence of identity} % (fold)

Several others have written about how privacy constraints impact personalized systems~\cite{Teltzrow2004,Kobsa2007}. In many ways, the very nature of anonymity is an extension of privacy. However, the absence of \emph{identity} presents an entirely different challenge.

There are ways of coping with the absence of user identification. Kobsa, for instance, describes an approach that makes heavy use of pseudonyms designed around this problem~\cite{Kobsa2003}.
However, appear.in is not only a pseudonymous service -- it is a fully anonymous service; there are no user accounts, and there is no login.

Tracking users, then, is obviously a bit of a struggle.

\subsubsection{Tracking users}

appear.in runs in the browser. The first time a user enters the site, a cookie is set with a random value uniquely identifying the browser over time. This value is sent with every event to the instrumentation service used for user analytics.

Unfortunately, using only cookies for identification has its clear downsides. Should the user clear the browser cache, switch browsers, use the browser ``incognito''\footnote{Google lingo for private browsing, where previously set cookies are not available (among other things).}, or simply use multiple machines, then different user ids will be generated for each case.

Without tracking more user information -- IPs and locations, for instance -- there is no easy way of tying these user ids together\footnote{Some suggestions on how to solve this is found in the suggestions for further work, in section~\ref{sec:further_work}.}, and to reason about them as a single user. However, collecting this information about users goes against appear.in's privacy policy, and it has been deemed more interesting to see what can be done without crossing this line.

\section{User adaptation}
\label{sec:adaptation_intro}

Before answering the question of \emph{how} to do user adaptation, let's look into the more fundamental issue of \emph{why}. Why would we want to adapt a product to its users?

There can be several reasons. Dijkstra once described a way in which humans and computers differ with a short anecdote involving piano playing and his mother~\cite{Dijkstra1982}:

\begin{quote}
    To end up my talk I would like to tell you a small story, that taught me the absolute mystery of human communication. I once went to the piano with the intention to play a Mozart sonata, but at the keyboard I suddenly changed my mind and started playing Schubert instead. After the first few bars my surprised mother interrupted me with ``I thought you were going to play Mozart!''. She was reading and had only seen me going to the piano through the corner of her eye. It then transpired that, whenever I went to the piano, she always knew what I was going to play! How? Well, she knew me for seventeen years, that is the only explanation you are going to get.
\end{quote}

Humans constantly monitor the world around them, and model the objects and people in it. The kind of implicit modeling and predictive behavior is something computers generally have a hard time dealing with.

In addition to this, the digital world around us is becoming more and more complex. This complexity will at some point overwhelm users. Adapting to the user's expectations may help in this~\cite{Vrieze}.

For a survey of the field of user adaptation, see section~\ref{sec:user_adaptation_survey}.

\subsection{A new context}
\label{sub:adaptation_context_intro}

The application case of appear.in presents a few novel qualities for an adaptive system to deal with. Firstly, we want to do feature-level user adaptation of a web application, and secondly, we want to do so with extremely challenging privacy constraints.

The first point, what being an \emph{application} running on the web entails for the relevance of previous research, is discussed in depth in section~\ref{sub:traditional_web_personalization}.

The second point, the impact of tight privacy constraints, requires some discussion before we take on the concrete problem specification of this thesis.

As we shall see throughout chapter~\ref{Chapter2}, most of the surveyed adaptation systems have a notion of \emph{users}, who can be tracked over time. Furthermore, most traditional user modeling systems even assume either the availability of demographic information or some notion of \emph{consumption of resources}, which in turn both can aid in constructing user models.

We, however, do not enjoy such luxuries. A large part of this thesis will look at how far we can get with only behavioral data, when even identity is at best an unstable notion.

Section~\ref{sec:privacy_vs_personalization_survey} will take a closer look at previous research on the notion of ``privacy versus personalization''.

\section{Problem specification}
\label{sec:problem_specification}



@TODO: Pull together above sections, explain exact case for this project. How does it differ from the general case?

\textbf{Main research question:} Can users of anonymous video conferencing services be clearly divided into user classes based on their behavior, and if so, to what effect can personalization improve their activity level?

\begin{enumerate}
  \item Are users of video conferencing services such as appear.in clearly dividable into separate groups based only on their behavior within the service? Do these patterns reflect those seen elsewhere -- in other types of internet services or even in real life?
  \item Is it feasible to personalize treatments to these user classes? Does it stimulate users into becoming more active users?
  \begin{enumerate}
    \item Is this something these users want?
    \item Do the inferred preferences of the detected user classes significantly differ from each other?
    \item Can the personalized treatments be devised in such a way as to stimulate the moving of users in the direction of any desired user class?
  \end{enumerate}
  \item How can a toolkit be devised to handle the following?
  \begin{enumerate}
    \item User classification based on behavior.
    \item Product personalization based on a relevant user's class.
    \item Tracking of each treatment's effect on each user class.
    \item Prioritize using the most effective treatments without introducing statistical bias (see multi-armed bandit).
    \item Allow product developers to easily access results to improve future feature prioritization.
  \end{enumerate}
\end{enumerate}

\section{Organization of the thesis}
\label{sec:thesis_organization}

This paper is organized as follows.

Chapter~\ref{Chapter2} surveys relevant literature, similar applications, provides an in-depth analysis of the available data.
Chapter~\ref{Chapter3} describes the system design and the reasoning behind central design choices.
In chapter~\ref{Chapter4} we'll look at the execution results, and see how they evaluate.
Chapter~\ref{Chapter5} summarizes the most important takeaways, and suggests further work.

%----------------------------------------------------------------------------------------
