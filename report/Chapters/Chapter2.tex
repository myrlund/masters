\chapter{Survey}

\label{Chapter2}

\lhead{Chapter 2. \emph{Survey}}

\section{Similar problem domains} % (fold)
\label{sec:similar_problem_domains}

Although the case in question is a specific service, the techniques in use and the limitations needing to be dealt with should apply to many kinds of applications.

When demographics are absent and identity is highly unreliable, this will unevitably lead to some highly sparse usage data and quite a bit of noise in the user modeling data.
Assuming that the usage patterns of the users actually differ, will these patterns be clearly and reliably identifiable?
Will it be possible to base an adaptive personalization system on this kind of data?
When using an anonymous system, will users welcome a personalized product, or will they regard this as breaking with their perceived concept of anonymity?

These are all questions that apply especially to appear.in, but that may also apply to many other anonymous, web-based systems.

% section similar_problem_domains (end)

\section{Similar research} % (fold)
\label{sec:similar_applications}

The approach taken to adaptive personalization is based heavily on the work by Vrieze in ``Fundaments of Adaptive Personalization''~\cite{Vrieze}.

\subsection{Relevant literature} % (fold)
\label{sub:relevant_literature}

Teltzrow and Kobsa's work on privacy-driven personalization systems provides insight into a lot of the issues surrounding the lack of demographic information in personalization~\cite{Teltzrow2004,Kobsa2007}. However, the main focus of their work seems to be that users are more willing to provide demographic information if that information is not backtracable to themselves, through pseudonymous personalization.
This matches the application in question quite poorly, as the goal is not to obtain demographics -- rather to attempt to cope without it.

\emph{@TODO} Explore and include literature on:

\begin{itemize}
  \item Clustering and segmenting users, choice of algorithms etc.
  \item Visualizing and evaluating clusters.
  \item A/B testing, multivariate testing, multi-arm bandits.
  \item Motivations for adaptive personalization. Online business models?
  \item Anonymity: do users experience personalization as an overstep?
\end{itemize}

% subsection relevant_literature (end)

% section similar_applications (end)



