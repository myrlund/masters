%%%%%%%%%%%%%%%%%%%%%%%%%%%%%%%%%%%%%%%%%
% Masters/Doctoral Thesis
% LaTeX Template
% Version 1.41 (9/9/13)
%%%%%%%%%%%%%%%%%%%%%%%%%%%%%%%%%%%%%%%%%

%----------------------------------------------------------------------------------------
%	PACKAGES AND OTHER DOCUMENT CONFIGURATIONS
%----------------------------------------------------------------------------------------

\documentclass[11pt, a4paper, oneside]{Thesis} % Paper size, default font size and one-sided paper

\usepackage[square, numbers, comma, sort&compress]{natbib} % Use the natbib reference package - read up on this to edit the reference style; if you want text (e.g. Smith et al., 2012) for the in-text references (instead of numbers), remove 'numbers'

\usepackage{minted}

\usepackage{amsthm}
\newtheorem{hypothesis}{Hypothesis}
\newtheorem{persona}{Persona}

\usepackage{verbatim}

%----------------------------------------------------------------------------------------
%	DOCUMENT VARIABLES
%	Fill in the lines below to update the thesis template
%	If you wish to cite each of the variables defined below, look at the
%	section above for the citation command e.g. \examiner{} below is
%	defined as \examname above so you cite it as \examname
%----------------------------------------------------------------------------------------

\thesistitle{User adaptation in \\ anonymous web applications}
\supervisor{Prof. Heri \textsc{Ramampiaro} \\Prof. Helge \textsc{Langseth}}
\examiner{}
\degree{Master of Science in Computer Science}
\authors{Jonas \textsc{Myrlund}}


\addresses{}
\subject{}
\keywords{}

\university{Norwegian University of Science and Technology}
\UNIVERSITY{NORWEGIAN UNIVERSITY OF SCIENCE AND TECHNOLOGY}
       
\department{Department of Computer and Information Science}
\DEPARTMENT{DEPARTMENT OF COMPUTER AND INFORMATION SCIENCE}

\group{Intelligent Systems Group}
\GROUP{INTELLIGENT SYSTEMS GROUP}

\faculty{Faculty of Information Technology, Mathematics and Electrical Engineering}
\FACULTY{FACULTY OF INFORMATION TECHNOLOGY, MATHEMATICS AND ELECTRICAL ENGINEERING}


\graphicspath{{Figures/}} % Specifies the directory where pictures are stored

\hypersetup{urlcolor=blue, colorlinks=true} % Colors hyperlinks in blue - change to black if annoying
\title{\ttitle} % Defines the thesis title - don't touch this


\DeclareMathOperator*{\argmax}{\arg\!\max}

\begin{document}


\frontmatter % Use roman page numbering style (i, ii, iii, iv...) for the pre-content pages

\setstretch{1.3} % Line spacing of 1.3

% Define the page headers using the FancyHdr package and set up for one-sided printing
\fancyhead{} % Clears all page headers and footers
\rhead{\thepage} % Sets the right side header to show the page number
\lhead{} % Clears the left side page header

\pagestyle{fancy} % Finally, use the "fancy" page style to implement the FancyHdr headers

\newcommand{\HRule}{\rule{\linewidth}{0.5mm}} % New command to make the lines in the title page

% PDF meta-data
\hypersetup{pdftitle={\ttitle}}
\hypersetup{pdfsubject=\subjectname}
\hypersetup{pdfauthor=\authornames}
\hypersetup{pdfkeywords=\keywordnames}

%----------------------------------------------------------------------------------------
%	TITLE PAGE
%----------------------------------------------------------------------------------------

\begin{titlepage}
\begin{center}

\newcommand{\thesistype}{Master's thesis, spring 2014}

\textsc{\LARGE \univname}\\[1.5cm] % University name
\textsc{\Large \thesistype}\\[0.5cm] % Thesis type

\HRule \\[0.4cm] % Horizontal line
{\huge \bfseries \ttitle}\\[0.4cm] % Thesis title
\HRule \\[1.5cm] % Horizontal line
 
\begin{minipage}{0.4\textwidth}
\begin{flushleft} \large
\emph{Author:}\\
\authornames \\ ~ % Author name - remove the \href bracket to remove the link
\end{flushleft}
\end{minipage}
\begin{minipage}{0.4\textwidth}
\begin{flushright} \large
\emph{Supervisors:} \\
\supname % Supervisor name - remove the \href bracket to remove the link  
\end{flushright}
\end{minipage}\\[3cm]
 
% \large \textit{A thesis submitted in fulfilment of the requirements\\ for the degree of \degreename}\\[0.3cm] % University requirement text
% \textit{in the}\\[0.4cm]
% \groupname\\\deptname\\[2cm] % Research group name and department name
 
{\large \today}\\[4cm] % Date
%\includegraphics{Logo} % University/department logo - uncomment to place it
 
\vfill
\end{center}

\end{titlepage}


% %----------------------------------------------------------------------------------------
% %  CHANGELOG
% %----------------------------------------------------------------------------------------
%
% \clearpage % Start a new page
%
% \lhead{\emph{Changelog}} % Set the left side page header to "Physical Constants"
%
% \changelog
%
% The latest changes are listed first.
%
% \begin{description}
%   \input{log.txt}
% \end{description}
%
% \clearpage

%----------------------------------------------------------------------------------------
%  QUOTATION PAGE
%----------------------------------------------------------------------------------------

\pagestyle{empty} % No headers or footers for the following pages

\null\vfill % Add some space to move the quote down the page a bit

\textit{\large{``C is for cookie, and cookie is for me; C is for cookie, and cookie is for me.''}}

\begin{flushright}
Cookie Monster, Sesame Street
\end{flushright}

\vfill\vfill\vfill\vfill\vfill\vfill\null % Add some space at the bottom to position the quote just right

\clearpage % Start a new page

%----------------------------------------------------------------------------------------
%	ABSTRACT PAGE
%----------------------------------------------------------------------------------------

\addtotoc{Abstract} % Add the "Abstract" page entry to the Contents

\abstract{\addtocontents{toc}{\vspace{1em}} % Add a gap in the Contents, for aesthetics

The goal of the project in this thesis is to explore the viability of an approach to user adaptation where the application context is significantly more constraining than in most cases seen in previous academic work.

In this project I describe a system for rolling out product features incrementally in an optimal way, based on feature adoption statistics within user segments. In other words, the described system allows for simple personalization of the product while experimenting with different feature variations.

Identifying how different classes of users use an application differently can be useful on several levels, and it is often the case that some are more desirable than others. This can be due to its associated users generating more revenue, using the product more, inviting their friends, or similar. This project describes a system capable of not only identifying user classes based on user behavior, but more importantly: a framework for identifying the most effective ways of adapting the product to these user class segments, in effect driving users in a desirable direction.

% More specifically, given a set of identified user classes and a set of predefined treatments, we want to find out how each treatment affects each user class. Although the project implementation will specifically target the video conferencing service appear.in, a major research question will be to what extent the results generalize.

We find that there are indeed clear differences in feature adoption across the identified user segments. However, due to uncertainty caused by domain constraints, it is uncertain to what extent the results generalize.

}

\clearpage % Start a new page

% %----------------------------------------------------------------------------------------
% %  ACKNOWLEDGEMENTS
% %----------------------------------------------------------------------------------------
%
\setstretch{1.3} % Reset the line-spacing to 1.3 for body text (if it has changed)

\acknowledgements{\addtocontents{toc}{\vspace{1em}} % Add a gap in the Contents, for aesthetics

First and foremost, I would like to thank my supervisors, Heri Ramampiaro and Helge Langseth. While providing me with plenty of space to do things my own way, they have been an immense help in finding the academic angle for the project, and in continuously steering me back on course every time I've lost track of the goal.

Telenor Digital AS have allowed me to write this thesis with them, and for this I am very grateful. Especially big thanks to Svein Yngvar Willassen, my supervisor at Telenor Digital, who both talked me into doing a master's project with appear.in, and assisted me greatly along the way.

Obviously, a big thanks to the entire appear.in team. It is extremely inspiring to do research when you thoroughly believe in the application you are working to improve. Ahead of me throughout the entire project period, they have improved the application immensely and attracted huge amounts of users who generated lots of data -- without which this thesis would have been impossible to write!

}
\clearpage % Start a new page

%----------------------------------------------------------------------------------------
%	LIST OF CONTENTS/FIGURES/TABLES PAGES
%----------------------------------------------------------------------------------------

\pagestyle{fancy} % The page style headers have been "empty" all this time, now use the "fancy" headers as defined before to bring them back

\lhead{\emph{Contents}} % Set the left side page header to "Contents"
\tableofcontents % Write out the Table of Contents

\lhead{\emph{List of Figures}} % Set the left side page header to "List of Figures"
\listoffigures % Write out the List of Figures

\lhead{\emph{List of Tables}} % Set the left side page header to "List of Tables"
\listoftables % Write out the List of Tables

% %----------------------------------------------------------------------------------------
% %	ABBREVIATIONS
% %----------------------------------------------------------------------------------------
%
% \clearpage % Start a new page
%
% \setstretch{1.5} % Set the line spacing to 1.5, this makes the following tables easier to read
%
% \lhead{\emph{Abbreviations}} % Set the left side page header to "Abbreviations"
% \listofsymbols{ll} % Include a list of Abbreviations (a table of two columns)
% {
% % \textbf{API} & \textbf{A}pplication \textbf{P}rogramming \textbf{I}nterface \\
% }


% %----------------------------------------------------------------------------------------
% %  SYMBOLS
% %----------------------------------------------------------------------------------------
%
% \clearpage % Start a new page
%
% \lhead{\emph{Symbols}} % Set the left side page header to "Symbols"
%
% \listofnomenclature{lll} % Include a list of Symbols (a three column table)
% {
% $a$ & distance & m \\
% $P$ & power & W (Js$^{-1}$) \\
% % Symbol & Name & Unit \\
%
% & & \\ % Gap to separate the Roman symbols from the Greek
%
% $\omega$ & angular frequency & rads$^{-1}$ \\
% % Symbol & Name & Unit \\
% }

% %----------------------------------------------------------------------------------------
% %  DEDICATION
% %----------------------------------------------------------------------------------------

%
% \setstretch{1.3} % Return the line spacing back to 1.3
%
% \pagestyle{empty} % Page style needs to be empty for this page
%
% \dedicatory{For/Dedicated to/To my\ldots} % Dedication text
%
% \addtocontents{toc}{\vspace{2em}} % Add a gap in the Contents, for aesthetics

%----------------------------------------------------------------------------------------
%	THESIS CONTENT - CHAPTERS
%----------------------------------------------------------------------------------------

\mainmatter % Begin numeric (1,2,3...) page numbering

\pagestyle{fancy} % Return the page headers back to the "fancy" style

% Include the chapters of the thesis as separate files from the Chapters folder
% Uncomment the lines as you write the chapters

\chapter{Introduction}

\label{Chapter1}

\lhead{Chapter 1. \emph{Introduction}}

\section{Background and motivation}
\label{intro:background_and_motivation}

  The advent of HTML5 enables us to build increasingly sophisticated applications that run right in the web browser on demand, as we will discuss further in~\ref{survey:modern_web}.

  Apart from the tendency to move ever closer to feature parity with native applications, most web applications share a common trait seldom seen elsewhere: they are available instantly and on-demand. Although some applications require the user to sign up or in other ways get past a paywall, the notion of a web application as being available \emph{without installation} remains.

  This lack of friction is something many web applications leverage. Indeed, the main competition among applications is often a matter of minimizing friction: a push towards simplicity and ease-of-use. This often involves the absence of authentication. As the users and the legislative forces governing the Internet are becoming more and more privacy-aware, there is little reason to believe this application niche is going away in the near future (see~\ref{survey:online_identity} for discussion).

  A real-world application adhering to the concept of minimizing friction is the application case for this project; appear.in\footnote{Available at \url{https://appear.in}.} is an attempt at providing a full-fledged video conferencing service with as little friction as possible. Among other things, this entails the application having no particular notion of users, and with that a highly unstable notion of \emph{identity} -- as discussed in~\ref{survey:identity}.

  An interesting question is, then, what can we do with the unstable data situation at hand? Is the behavioral data we have available enough to be able to do any significant user adaptations, for instance? In very broad terms, this is what this project set out to answer.

  Before moving to the concrete research questions, let us have a look at ways of dealing with privacy issues surrounding the handling of identity in user adaptation settings.

  \subsection{User adaptation versus privacy}
  \label{intro:adaptation_vs_privacy}

    Internet privacy entails the protection of both personally identifying information (PII), as well as non-PII -- such as a site visitor's behavior on a website. To some extent, both types of information are needed to provide any sort of user adaptation.

    Obrenović and den Haag survey 4 viable ways of approaching the concept of identity when doing any sort of user customization or user adaptation~\cite{Obrenovic2012}. The approach taken in this project is, strictly out of necessity, an adapted version of their least intrusive identity management pattern, the masked-external-user pattern. See~\ref{survey:masked_external_user_pattern} for a thorough explanation of the pattern and the particular adaptations required for this application case.

\section{Research questions}
\label{intro:research_questions}

  This project will investigate whether users of a simple service, like the appplication case, fall into clear sterotypical patterns. Further, it will attempt to measure to what extent these user stereotypes can be used as a basis for user adaptations.

  Since these problems in many ways build on each other, they will have to be answered in sequence. More specifically, I will attempt to answer the following questions:

  \begin{enumerate}
    \item Is it possible to consistently stereotype the users of simple web applications without requiring explicit authentication?
    \item Are these stereotypes usable as a basis for user adaptations within the application?
  \end{enumerate}

  The problem of stereotyping users will involve generating user models. However, as the application does not have a clear notion of a user, much of the discussion will be dedicated to dealing with the ways we can and cannot circumvent this problem.

\section{Organization of the thesis}
\label{intro:thesis_organization}

  This paper is organized as follows.

  Chapter~\ref{Chapter2} surveys the application case and the available data, discusses some important identification issues, and surveys relevant research.
  Chapter~\ref{Chapter3} describes the approach taken to answer the research questions.
  In chapter~\ref{Chapter4} an implementation of the approach described in chapter~\ref{Chapter3} is evaluated, before chapter~\ref{Chapter5} discusses some important takeaways, and suggests further work.

\chapter{Survey}

\label{Chapter2}

\lhead{Chapter 2. \emph{Survey}}

\section{Similar problem domains} % (fold)
\label{sec:similar_problem_domains}

Although the case in question is a specific service, the techniques in use and the limitations needing to be dealt with should apply to many kinds of applications.

When demographics are absent and identity is highly unreliable, this will unevitably lead to some highly sparse usage data and quite a bit of noise in the user modeling data.
Assuming that the usage patterns of the users actually differ, will these patterns be clearly and reliably identifiable?
Will it be possible to base an adaptive personalization system on this kind of data?
When using an anonymous system, will users welcome a personalized product, or will they regard this as breaking with their perceived concept of anonymity?

These are all questions that apply especially to appear.in, but that may also apply to many other anonymous, web-based systems.

% section similar_problem_domains (end)

\section{Similar research} % (fold)
\label{sec:similar_applications}

The approach taken to adaptive personalization is based heavily on the work by Vrieze in ``Fundaments of Adaptive Personalization''~\cite{Vrieze}.

\subsection{Relevant literature} % (fold)
\label{sub:relevant_literature}

Teltzrow and Kobsa's work on privacy-driven personalization systems provides insight into a lot of the issues surrounding the lack of demographic information in personalization~\cite{Teltzrow2004,Kobsa2007}. However, the main focus of their work seems to be that users are more willing to provide demographic information if that information is not backtracable to themselves, through pseudonymous personalization.
This matches the application in question quite poorly, as the goal is not to obtain demographics -- rather to attempt to cope without it.

\emph{@TODO} Explore and include literature on:

\begin{itemize}
  \item Clustering and segmenting users, choice of algorithms etc.
  \item Visualizing and evaluating clusters.
  \item A/B testing, multivariate testing, multi-arm bandits.
  \item Motivations for adaptive personalization. Online business models?
  \item Anonymity: do users experience personalization as an overstep?
\end{itemize}

% subsection relevant_literature (end)

% section similar_applications (end)




\chapter{Approach}

\label{Chapter3}

\lhead{Chapter 3. \emph{Approach}}

\section{System overview} % (fold)
\label{sec:system_overview}

The proposed system is constructed around the data flowing through it.

The system takes the data stepwise through an ingestion pipeline, importing, filtering and cleaning it, before churning it into user models. This particular part of the system is elaborated on further in section~\ref{sec:data_ingestion_and_preprocessing}.

Once user models are in place, the system generates user segments. These are stored in a database for future use. There are several viable approaches to the segmentation task. These are discussed in more detail in section~\ref{sec:user_modeling_and_clustering}.

As it turns out, when designing completely autonomous adaptative systems, we need more data than user segments to effectively drive the adaptive component. When considering whether to enable a feature or an interface switch, we need to know whether doing so will be advantageous to the user segment in question. This is discussed in section~\ref{sec:adaptation_component}.

% section system_overview (end)

\section{Data ingestion and preprocessing} % (fold)
\label{sec:data_ingestion_and_preprocessing}

Before the interesting parts of the system can start doing their work, the data needs to be transformed from \emph{a series of chronological raw events} to \emph{a set of user models}.

The amount of data can be arbitrarily sizable, and will grow linearly with user activity. The system architecture has been designed to be able to cope with this; its functional and data-driven nature should be easily adaptable to hugely scalable programming paradigms like MapReduce.

\begin{figure}[h]
  \centering
    \includegraphics[width=\textwidth]{Figures/ingestion-pipeline}
  \caption{The ingestion pipeline broken into 4 steps. The color of each node indicates means of storage: \emph{Blue} indicates a RDBMS, \emph{green} indicates a graph database, whereas \emph{gray} is used to indicate flat file storage.}
  \label{fig:ingestion-pipeline}
\end{figure}

\subsection{Generating the raw data}
\label{sub:generating_data}

The system input is a chronological series of raw events sent from the production system.

These are instrumented via an external analysis service called KISSmetrics\footnote{\url{https://www.kissmetrics.com/}}. It is a user analysis system designed around tracking individual users' behavior. It works by calling their REST API with the following data:

\begin{enumerate}
  \item person identifier
  \item event name
  \item user properties
\end{enumerate}

The consistency of the personal identifier has already been discussed extensively in the introductory chapters, especially section~\ref{sub:anonymity_privacy}, but a short technical introduction to the actual production system is in order.

\subsection{Event instrumentation in KISSmetrics}
\label{sub:event_instrumentation}

To enable effective utilization of the KISSmetrics instrumentation functionality, they supply a client library for the purpose. This client library handles a few central things for us:

\begin{enumerate}
  \item Person identity storage and loading over subsequent page loads.
  \item The low-level instrumentation of events.
  \item Simple A/B testing facilities.
\end{enumerate}

When the KISSmetrics client library is loaded, the person identity is automatically either retrieved from the browser cookies, or generated.

The identity of a person is a unique randomly generated string, which serves no other purpose than to track the identity of the browser over time. No personal data is stored, nor is it available to us.

Whenever something ``interesting'' happens, an event is sent to the KISSmetrics instrumentation service. An ``interesting'' event is anything that tells us about how the users use the service, both in terms of general activity and in terms of feature adoption. Every event is tagged with the person identity, as well as an event name and a timestamp.

The KISSmetrics service provides several analytical tools to dig into this data, thereamongst funnel reports and cohort reports -- as depicted in figure~\ref{fig:funnel-report} and figure~\ref{fig:cohort-report}.

\begin{figure}[h]
  \centering
    \includegraphics[width=\textwidth]{Figures/screenshots/km/funnel-example}
    \caption{Example of a simple funnel report.}
    \label{fig:funnel-report}
\end{figure}

\begin{figure}[h]
  \centering
    \includegraphics[width=\textwidth]{Figures/screenshots/km/cohort-example}
    \caption{Example of a cohort report.}
    \label{fig:cohort-report}
\end{figure}

% section data_ingestion_and_preprocessing (end)

\section{User modeling and clustering} % (fold)
\label{sec:user_modeling_and_clustering}


Further, whenever

% section user_modeling_and_clustering (end)

\section{Adaptation component} % (fold)
\label{sec:adaptation_component}

\subsection{Evaluation metrics} % (fold)
\label{sub:evaluation_metrics}

\subsubsection{How to measure a positive user experience} % (fold)

% section evaluation_metrics (end)

\subsection{Applying the personalized feature set} % (fold)
\label{sub:applying_the_personalized_feature_set}

Description of the FlagService model.

% section applying_the_personalized_feature_set (end)

\subsection{Tracking user treatments} % (fold)
\label{sub:tracking_user_treatments}

% section tracking_user_treatments (end)

\subsection{Visualizing effects} % (fold)
\label{sub:visualizing_effects}

% section visualizing_effects (end)

% section adaptation_component (end)

\subsection{Differentiating product features} % (fold)
\label{sec:differentiating_product_features}

% section differentiating_product_features (end)

\section{Evolving the user models} % (fold)
\label{sec:evolving_the_user_models}

\subsection{Multi-arm bandits}

\subsection{Tracking individual treatment}

% section evolving_the_user_models (end)

\section{Visualization requirements (?)} % (fold)
\label{sec:visualization_requirements}

% section visualization_requirements (end)

\chapter{Evaluation}

\label{Chapter4}

\lhead{Chapter 4. \emph{Evaluation}}

\section{Description of the evaluated dataset} % (fold)
\label{sec:description_of_the_evaluated_dataset}

% section description_of_the_evaluated_dataset (end)

\section{Prerequisites} % (fold)
\label{sec:prerequisites}

For any of this user adaptation to have any actual use, we will need to find out whether there is actually significant variation in feature adoption between clusters; it matters little whether users exhibit differing behavior with regard to existing functionality, if this behavior yields no basis for predicting feature adoption in the future. We need an inductive bias.

An easy way of discovering whether this inductive bias is present in the user base is to simply log whether there is significant variance across the user groups in their users' adoption of new features. Thus, this question of whether the predictive bias actually exists will be included as a central part of the actual experiment itself, and be thoroughly discussed through the next sections.

% section prerequisites (end)

\section{Experimental setup} % (fold)
\label{sec:experimental_setup}

\emph{Condensed version.}

The main experiment determines whether an alteration of the service affects different kinds of users differently.
More specifically, do users in clusters $C_1, \ldots, C_n$ employ function $f$ in significantly differing ways? We find out without touching a large percentage of the users, and can use the results to individually enable features for users that

We break the process into three steps, each elaborated on in~\ref{sub:clustering_of_users}, \ref{sub:a_b_testing_features}, and \ref{sub:evaluating_the_results}:

\begin{enumerate}
  \item A/B test feature $f$ on a percentage of all users.
  \item Separate users into clusters $C_1, \ldots, C_n$.
  \item Analyze results: did the test results vary significantly between clusters?
\end{enumerate}

If the final answer is ``yes'', we can store our results and use them as a basis for adaptating the interface for each user.

\subsection{A/B testing features} % (fold)
\label{sub:a_b_testing_features}

\emph{@TODO: Brief description of how to fit the actual testing regime into the production system.}

% subsection a_b_testing_features (end)

\subsection{User clustering} % (fold)
\label{sub:clustering_of_users}

\emph{@TODO: Describe methods and algorithms used. What kind of results are expected?}

% subsection clustering_of_users (end)


% section experimental_setup (end)

\section{Adapting the user interface} % (fold)
\label{sec:adapting_the_user_interface}

The adaptation component consists of two important steps.

\begin{enumerate}
  \item Determine which type of user (ie. \emph{cluster}) will have the most to gain from having feature $f$ enabled.
  \item Tagging the users with cluster and enabling features for
\end{enumerate}

\subsection{Evaluating the results} % (fold)
\label{sub:evaluating_the_results}

Seeing as the application does not yield any direct and real payoff, we will evaluate the performance of each variation relatively, using some general key metrics. This approach is described in~\cite{Yue2012}.

For appear.in, the most important key metrics are ``time on site'' and ``number of visits in the last $n$ days''. These are combined in providing a relative success metric for each variation.

\emph{@TODO: Verify key metrics.}

% subsection evaluating_the_results (end)


% section adapting_the_interface (end)

\section{Results} % (fold)
\label{sec:results}

% section results (end)

\section{Discussion} % (fold)
\label{sec:discussion}

% section discussion (end)

\chapter{Conclusion}

\label{Chapter5}

\lhead{Chapter 5. \emph{Conclusion}}

% \emph{I konklusjonen din, beskriv status for ditt arbeid. Oppsummer hva du har oppnådd, sammenlignet med hva du opprinnelig ønsket å oppnå. Relater arbeidet til tidligere relevant arbeid. Foreslår videre arbeid som du tror vil være verdt.}

\section{Summary} % (fold)
\label{conclusion:sec:summary}

As shown in the previous chapter, we find that there are indeed clear differences in feature adoption across the identified user segments.

The domain constraints and the resulting scarcity and unreliability of user modeling data makes for clustering results that are at best hard to intuitively understand. This hinders us in reasoning about the various ways of applying the user adaptations, ie. the ways of distributing the feature variants among the identified user segments.

% Often, this process is based on more diffuse concepts, like business intelligence or intuition about the users.

The controlled experiments of section~\ref{approach:feature_experiments} provide us with hard evidence of behavioral correlations in our user base -- or indeed, lack thereof. This disentangles us from the error-prone decision making otherwise involved in setting up group adaptations.

\section{Generalizing the system} % (fold)
\label{conclusion:sec:generalizing_the_system}

Section~\ref{survey:sub:generality} analyzes the application case, and identifies the two main challenges of providing user adaptation:

\begin{enumerate}
  \item the absence of demographics
  \item the unreliability of user identity
\end{enumerate}

Together, these two factors comprise an unusually harsh environment for an adaptative system to operate in. In ... we predicted that the clusters would be hard to intuitively understand, as they would effectively represent a noisy and incomplete version of the actual user base.

As we saw in chapter~\ref{Chapter4}, these predictions did indeed match the results.

Due to uncertainty caused by domain constraints, it is hard to tell to what extent the results generalize.

% section generalizing_the_system (end)

\section{Suggestions for further work} % (fold)
\label{conclusion:sec:further_work}

The work presented can be improved in many ways.

Within the domain of appear.in, and applications like it, the event data cleaning and user modeling processes have a lot to gain.

Firstly, it would be interesting to see if similar results are achieved for other applications.

Keep cluster centroids to enable ``classification'' of new users into groups whose behavior is already known.


%----------------------------------------------------------------------------------------
%	THESIS CONTENT - APPENDICES
%----------------------------------------------------------------------------------------

\addtocontents{toc}{\vspace{2em}} % Add a gap in the Contents, for aesthetics

\appendix % Cue to tell LaTeX that the following 'chapters' are Appendices

% Include the appendices of the thesis as separate files from the Appendices folder
% Uncomment the lines as you write the Appendices

\chapter{Evaluation Results}

\label{AppendixA}

\lhead{Appendix A. \emph{Evaluation Results}}



\chapter{Design Documents}

\label{AppendixB}

\lhead{Appendix B. \emph{Design Documents}}


% \chapter{Evaluation Results}

\label{AppendixC}

\lhead{Appendix C. \emph{Evaluation Results}}



\addtocontents{toc}{\vspace{2em}} % Add a gap in the Contents, for aesthetics

\backmatter

%----------------------------------------------------------------------------------------
%	BIBLIOGRAPHY
%----------------------------------------------------------------------------------------

\label{Bibliography}

\lhead{\emph{Bibliography}} % Change the page header to say "Bibliography"

\bibliographystyle{unsrtnat} % Use the "unsrtnat" BibTeX style for formatting the Bibliography

\bibliography{CustomBibliography,Bibliography}

\end{document}
